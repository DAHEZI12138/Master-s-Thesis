% !Mode:: "TeX:UTF-8"

\chapter{总结与展望}

\section{总结}

随着对人脑认知的不断深入,以及EEG采集设备和分析算法的大幅发展,基于EEG的BCI技术已经在众多行业崭露头角。基于BCI系统的人脑状态与运动意图辨识,能够为人机交互和健康检测提供丰富的信息指导,为运动障碍患者的康复进程提供助力。本课题针对现有BCI设备存在的不足,设计了一款名为JS-AINS-40的全新BCI系统。该系统的硬件部分体积较小,使用方便,成本低廉,信号质量可靠并且通道数量较多,能够满足当前的BCI使用需求。该系统的软件部分分别利用神经架构搜索技术和无监督域适应算法克服了当前EEG解码算法面临的设计困难和泛化性能较低问题,提升了BCI系统的辨识性能。综上所述,本课题的主要工作内容如下:

本课题所设计的JS-AINS-40系统由上位机、受试者以及中间的采集模块共同组成。其中采集模块作为整个系统的硬件核心,包括EEG采集盒和配套的脑电极帽两部分。脑电极帽遵循“国际10-20系统”完成电极排布。EEG采集盒中安装了采集模块PCB、用来固定PCB的安装立柱、LED导光柱、设备启动按钮、用来传输数据和给设备供电的USB接口以及用来连接脑电极帽的航空插头。采集模块PCB上的电路被划分为虚拟串口通讯模块、数字电源模块、STM32H743IIT6主控模块、SPI隔离模块、电源隔离模块、ADS1299采集模块、模拟电源模块、采集前端与右腿驱动模块。最终设计实现的采集设备具有40个EEG电极,一个右腿驱动电极和一个参考电极,能够以250 Hz-1000 Hz的采样率进行同步采样,利用USB接口实现供电,并基于虚拟串口完成数据传输。配套设计的下位机软件和上位机软件保证了采集模块能够持续稳定的完成EEG采集工作,并实时保存、滤波以及可视化使用者的EEG数据,增加了JS-AINS-40系统的实用性。最终的性能验证结果表明,本课题所设计的EEG采集设备的电压测量误差在$\pm 10\%$以内、共模抑制比在80 dB以上、噪声电平低于6 $\mu$V,并且能够通过医疗电气设备EMC测试。上述结果均满足国标要求,证明所设计的EEG采集设备具备足够优秀的采集性能和运行稳定性,能够为JS-AINS-40系统提供数据支撑。

JS-AINS-40系统的上位机除了数据保存,滤波和可视化,还配套了EEG解码算法。常见的EEG范式中,SSVEP和ERP需要借助外界引导进行诱发,在具有较高辨识精度的同时限制了实际应用范围。因此,JS-AINS-40系统的上位机解码算法针对MI范式和被动范式分别进行了设计。针对当前基于被动范式EEG的深度学习辨识模型设计复杂,调参耗时,无法同时应对多种不同类型EEG信号的特点,创新性的将自适应机制、早停机制与基于梯度的神经架构搜索算法进行结合,实现了对人脑情绪和疲劳状态的成功辨识。搜索到的网络模型针对公开情绪三分类数据集的分类准确率为82.96\%,公开疲劳数据集的辨识准确率为84.63\%,在众多先进对比算法中也极具竞争力。这一结果证实了所提出解码算法在被动范式EEG数据上的可靠性。同时,该解码算法在JS-AINS-40系统所采集的情绪数据集上同样取得了78.50\%的辨识结果,该结果说明了JS-AINS-40系统在人脑状态辨识领域具备足够潜力。

在MI范式EEG的辨识任务中,已有的EEG解码算法通常无法应对同时跨被试、跨时段以及多类别的实际应用场景。由于MI范式的EEG数据样本数较少,样本间差异较大,不同想象动作的活跃脑区存在大范围重叠,导致了算法难以进行有效辨识。本课题针对这些问题,设计基于无监督域适应算法的人脑运动意图辨识框架。这一框架包含针对MI-EEG数据的特征提取器、动态注意力模块、基于伪标签的自训练算法以及迭代训练策略——提取不同想象动作的频域特征、迁移训练集中每名被试的关键信息、在两轮迭代训练中增大测试集类间方差并进行模型校准。最终的实验结果表明,该MI辨识框架能够在三个公开MI数据集上取得69.51\%、82.38\%,以及90.98\%的辨识精度,成功击败了当前的最优模型。同时,在利用JS-AINS-40系统采集的MI数据集上,该框架取得了70.08\%的分类准确率,进一步验证了JS-AINS-40系统在运动意图辨识上的优异性能。

\section{展望}

本课题所设计的JS-AINS-40系统,是为当前基于EEG的BCI系统落地与推广所提供的一种具体解决方案。对其未来发展,提出以下展望:(1) JS-AINS-40系统在被动范式EEG数据和MI范式EEG数据上表现优异,但测试场景仍然偏少。未来可以探索其在例如生产控制、游戏娱乐以及教学研究等领域下的应用,充分挖掘其使用潜力。(2) JS-AINS-40系统仍然具备一定的拓展潜力。在保留其结构简单,操作便捷和低成本优势的前提下,JS-AINS-40的EEG导联数量可以进行进一步增减,开发应对不同需求的系列产品。(3) JS-AINS-40系统的在线运行功能还处于测试阶段,等待未来进一步优化,以应对可能出现的在线使用需求。(4) 由于MI范式数据量较少,难以直接驱动大型深度学习模型,导致神经架构搜索算法在其上性能表现不佳。因此,除了使用域适应算法外,还可以考虑使用数据增强策略,弥补数据量缺点,将神经架构搜索算法迁移至MI数据。



